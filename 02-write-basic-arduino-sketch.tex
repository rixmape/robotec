\documentclass[sectionframe]{oxblue-beamer}

\title{Write Basic Arduino Sketch}
\author[Mape, R. N. R.]{Rixdon Niño R. Mape}
\institute{
    College of Science \\
    Computer Science and Information Technology Department \\
    Bicol University
}
\date{02 July 2024}

\begin{document}

\begin{frame}
\titlepage
\end{frame}

\section{Arduino IDE Tour}

\begin{frame}{Arduino Integrated Development Environment (IDE)}
\begin{itemize}
    \item Software used to write and upload code to Arduino boards
    \item Download from \url{https://www.arduino.cc/en/software}
    \item \textbf{Demo:} Tour of the Arduino IDE
\end{itemize}
\end{frame}

\section{Basic Sketch Structure}

\begin{frame}[fragile]{Blinking LED}
\begin{minted}{arduino}
void setup() {
    pinMode(13, OUTPUT);
}

void loop() {
    digitalWrite(13, HIGH);
    delay(1000);
    digitalWrite(13, LOW);
    delay(1000);
}
\end{minted}
\end{frame}

\section{Variables and Data Types}

\begin{frame}[fragile]{Variable Definition}
\begin{minted}{arduino}
int ledPin = 13;

void setup() {
    pinMode(ledPin, OUTPUT);
}

void loop() {
    digitalWrite(ledPin, HIGH);
    delay(1000);
    digitalWrite(ledPin, LOW);
    delay(1000);
}
\end{minted}
\end{frame}

\begin{frame}{Data Types}
\begin{table}
\footnotesize
\centering
\begin{tabular}{|c|c|c|}
\hline
\textbf{Numeric Type} & \textbf{Bytes} & \textbf{Range} \\ \hline
\texttt{bool} & 1 & true (1) or false (0) \\ \hline
\texttt{byte} & 1 & 0 to 255 \\ \hline
\texttt{int} & 2 & -32,768 to 32,767 \\ \hline
\texttt{unsigned int} & 2 & 0 to 65,535 \\ \hline
\texttt{long} & 4 & -2,147,483,648 to 2,147,483,647 \\ \hline
\texttt{unsigned long} & 4 & 0 to 4,294,967,295 \\ \hline
\texttt{float} & 4 & -3.4028235E+38 to 3.4028235E+38 \\ \hline
\end{tabular}
\end{table}
\end{frame}

\section{Serial Communication}

\begin{frame}[fragile]{Serial Communication}
\begin{minted}{arduino}
int ledPin = 13;

void setup() {
    pinMode(ledPin, OUTPUT);
    Serial.begin(9600);
}

void loop() {
    digitalWrite(ledPin, HIGH);
    Serial.println("LED ON");
    delay(1000);
    digitalWrite(ledPin, LOW);
    Serial.println("LED OFF");
    delay(1000);
}
\end{minted}
\end{frame}

\section{Mathematical Operations}

\begin{frame}[fragile]{Arithmetic Operators}
\begin{minted}{arduino}
int a = 5;
int b = 3;

int sum = a + b;                        // 8
int difference = a - b;                 // 2
int product = a * b;                    // 15
int quotient = a / b;                   // 1 (integer division)
int remainder = a % b;                  // 2

float fQuotient = (float)a / b;         // 1.66667
\end{minted}
\end{frame}

\begin{frame}[fragile]{Grouping and Precedence}
\begin{minted}{arduino}
int a = 5;
int b = 3;

int result = a + b * 2;                 // 11
int groupedResult = (a + b) * 2;        // 16

result = a + b * 2 / 3;                 // 6
groupedResult = (a + b) * 2 / 3;        // 4
\end{minted}
\end{frame}

\begin{frame}[fragile]{Equality and Relational Operators}
\begin{minted}{arduino}
int a = 5;
int b = 3;

bool isEqual = (a == b);                // false (0)
bool isNotEqual = (a != b);             // true (1)

bool isGreater = (a > b);               // true (1)
bool isLess = (a < b);                  // false (0)
bool isGreaterOrEqual = (a >= b);       // true (1)
bool isLessOrEqual = (a <= b);          // false (0)
\end{minted}
\end{frame}

\begin{frame}[fragile]{Logical Operators}
\begin{minted}{arduino}
int a = 5;
int b = 3;

bool andResult = (a > 3) && (b < 5);    // true (1)
bool orResult = (a > 3) || (b > 5);     // true (1)
bool notResult = !(a > 3);              // false (0)
\end{minted}
\end{frame}

\section{Control Structures}

\subsubsection{Conditional Statements}

\begin{frame}[fragile]{\texttt{if} Statement}
\begin{minted}{arduino}
int a = 5;
int b = 3;

if (a > b) {
    Serial.println("a is greater than b");
}
\end{minted}
\end{frame}

\begin{frame}[fragile]{\texttt{if-else} Statement}
\begin{minted}{arduino}
int a = 5;
int b = 3;

if (a > b) {
    Serial.println("a is greater than b");
} else {
    Serial.println("a is less than or equal to b");
}
\end{minted}
\end{frame}

\begin{frame}[fragile]{Nested \texttt{if-else} Statement}
\begin{minted}{arduino}
int a = 5;
int b = 3;

if (a > b) {
    Serial.println("a is greater than b");
} else {
    if (a < b) {
        Serial.println("a is less than b");
    } else {
        Serial.println("a is equal to b");
    }
}
\end{minted}
\end{frame}

\begin{frame}[fragile]{Nested \texttt{if-else} Statement (cont'd)}
\begin{minted}{arduino}
int a = 5;
int b = 3;

if (a > b) {
    Serial.println("a is greater than b");
} else if (a < b) {
    Serial.println("a is less than b");
} else {
    Serial.println("a is equal to b");
}
\end{minted}
\end{frame}

\begin{frame}[fragile]{Switch Multiple-Selection Statement}
\begin{minted}{arduino}
char direction = 'F';

switch (direction) {
    case 'F':
        // Code for forward movement
        break;
    case 'B':
        // Code for backward movement
        break;
    case 'R':
        // Code for right turn
        break;
    case 'L':
        // Code for left turn
        break;
    default:
        // Code for invalid direction
        break;
}
\end{minted}
\end{frame}

\subsubsection{Looping Statements}

\begin{frame}[fragile]{\texttt{for} Counter-Controlled Loop}
\begin{minted}{arduino}
for (int i = 0; i < 5; i++) {
    Serial.print("Iteration ");
    Serial.println(i);
}
\end{minted}
\end{frame}

\begin{frame}[fragile]{\texttt{while} Condition-Controlled Loop}
\begin{minted}{arduino}
int i = 0;

while (i < 5) {
    Serial.print("Iteration ");
    Serial.println(i);
    i++;
}
\end{minted}
\end{frame}

\begin{frame}[fragile]{\texttt{do-while} Loop}
\begin{minted}{arduino}
int i = 0;

do {
    Serial.print("Iteration ");
    Serial.println(i);
    i++;
} while (i < 5);
\end{minted}
\end{frame}

\begin{frame}[fragile]{\texttt{break} Statement}
\begin{minted}{arduino}
for (int i = 0; i < 10; i++) {
    if (i == 5) {
        break;
    }
    Serial.println(i);
}
\end{minted}
\end{frame}

\begin{frame}[fragile]{\texttt{continue} Statement}
\begin{minted}{arduino}
for (int i = 0; i < 10; i++) {
    if (i % 2 == 0) {
        continue;
    }
    Serial.println(i);
}
\end{minted}
\end{frame}

\end{document}
