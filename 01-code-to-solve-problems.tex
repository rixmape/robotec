\documentclass[sectionframe]{oxblue-beamer}

\title{Code to Solve Problems}
\author[Mape, R. N. R.]{Rixdon Niño R. Mape}
\institute{
    College of Science \\
    Computer Science and Information Technology Department \\
    Bicol University
}
\date{02 July 2024}

\begin{document}

\begin{frame}
\titlepage
\end{frame}

\section{Coding in Robotics}

\begin{frame}{What is a computer?}
\begin{itemize}
    \item A machine that receives, stores, transforms, and outputs data
    \item Data of all kinds: numbers, text, images, sounds, etc.
    \item Logical units: input, output, memory, arithmetic and logic unit (ALU), and central processing unit (CPU), and secondary storage unit.
    \item Computer without software is like a car without a driver.
\end{itemize}
\end{frame}

\begin{frame}{What is program?}
\begin{itemize}
    \item Instructions that let a computer perform a specific task
    \item Developed using a programming language
    \item Programming languages: Python, Java, C++, etc.
    \item \textbf{Demo:} C program that prints "Hello, World!"
    \item Compilation translates source code to machine code
    \item An integrated development environment (IDE) simplifies the development process
\end{itemize}
\end{frame}

\begin{frame}{How is coding used in robotics?}
\begin{itemize}
    \item Actuation
    \item Perception
    \item Decision-making
    \item Machine learning
    \item Human-robot interaction
    \item System integration
    \item Simulation
\end{itemize}
\end{frame}

\section{Problem Solving Framework}

\begin{frame}{Define the problem}
\begin{itemize}
    \item State the problem clearly to gain clear understanding of what is required for its solution.
    \item Eliminate unimportant details and focus on the core problem.
    \item \textbf{Sorting Problem:} \par\smallskip Design an autonomous robot capable of sorting packages by size and destination in a logistics warehouse. The robot should operate with high accuracy and efficiency, reducing manual sorting tasks and improving overall processing speed.
\end{itemize}
\end{frame}

\begin{frame}{Analyze the problem}
\begin{itemize}
    \item Identify the problem inputs, outputs, and constraints.
    \item If the problem is not defined and analyzed properly, you will solve the wrong problem.
    \item \textbf{Task:} Analyze the sorting problem.
\end{itemize}
\end{frame}

\begin{frame}{Develop a solution}
\begin{itemize}
    \item Requires developing an \textbf{algorithm} to solve the problem
    \item Use \textbf{top-down design} by first listing major steps
    \item Break down the major steps into smaller steps through \textbf{stepwise refinement}
    \item To verify the algorithm, perform a \textbf{desk check} by hand
    \item \textbf{Task:} Write an algorithm for the sorting problem. Then, break down the algorithm into smaller steps.
\end{itemize}
\end{frame}

\begin{frame}{Implement the solution}
\begin{itemize}
    \item Translate the algorithm using a programming language
    \item Use documentation to explain complex parts
    \item \textbf{Demo:} A Python program that simulates the solution to the sorting problem
\end{itemize}
\end{frame}

\begin{frame}{Test the solution}
\begin{itemize}
    \item Verify that the program works as expected
    \item Use different set of test data to ensure the program is robust
    \item Identify and fix any bugs
\end{itemize}
\end{frame}

\begin{frame}{Refine the solution}
\begin{itemize}
    \item Review the code for efficiency.
    \item Optimize the algorithms for better performance.
    \item Refactor for readability and maintainability.
\end{itemize}
\end{frame}

\section{Code Quality}

\begin{frame}{Readability}
\begin{itemize}
    \item Use a consistent coding style, including indentation, naming conventions, and commenting practices.
    \item Choose meaningful variable and function names that reflect their purpose and functionality.
    \item Break code into smaller, modular functions and classes.
\end{itemize}
\end{frame}

\begin{frame}{Performance}
\begin{itemize}
    \item Analyze and optimize algorithms to reduce time and resource complexity.
    \item Be cautious about memory usage and avoid memory leaks.
    \item Implement caching mechanisms to reduce redundant computations.
\end{itemize}
\end{frame}

\begin{frame}{Maintainability}
\begin{itemize}
    \item Regularly refactor code to eliminate duplication, improve organization, and simplify complex logic.
    \item Use version control systems like Git to track changes and collaborate with other developers effectively.
    \item Write clear and concise documentation, including inline comments and high-level project documentation.
\end{itemize}
\end{frame}

\begin{frame}{Reliability}
\begin{itemize}
    \item Implement robust error-handling mechanisms to gracefully handle exceptions and errors.
    \item To make sure your code is up to mark make sure you have all the tests you need from unit tests to integration tests to error scenarios.
    \item In distributed systems, introduce redundancy and failover mechanisms to ensure continuous service availability.
\end{itemize}
\end{frame}

\end{document}
